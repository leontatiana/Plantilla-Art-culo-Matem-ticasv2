%--------------------------------IMPORTANTE-------------------------------
%-------------------------------------------------------------------------
%	1. Se debe modificar aquello que este comentado de la siguiente manera
							%"~~~~~MODIFICAR~~~~~~~~"
%	2. Sólo eliminar y reemplazar aquello que diga explicitamente
							%"eliminar"
%	3. Leer MUY BIEN los comentarios

%Esta plantilla fue desarrollada por Tatiana León Zamora, cualquier duda, pueden contactarme: yessica.leon@mail.escuelaing.edu.co
%-------------------------------------------------------------------------
%-------------------------------------------------------------------------
\documentclass[12pt, twoside]{article} %La opción twoside permite poner encabezados a la derecha e izquierda de la hoja
\usepackage[utf8]{inputenc}
\usepackage[spanish]{babel}
%Los siguientes son los packages más básicos para la escritura del artículo
\usepackage{graphicx, subfigure, wrapfig, float}
\usepackage{amsmath, amssymb, amsfonts, amsthm}
\usepackage{caption, multicol, multirow, longtable}
\usepackage{url, cclicenses, lastpage, color, booktabs}
\usepackage{graphicx,lipsum,wrapfig} %lipsum es un generador de texto dummy
\usepackage{fancyhdr, fancybox}
\usepackage{array, cite, enumerate} 
\usepackage{vmargin}
\usepackage{xcolor}

%Define comandos para cuando se quiera poner un teorema, lema, ejemplo, corolario o definición
\newtheorem{theo}{Teorema}
\newtheorem{examp}{Ejemplo}
\newtheorem{coro}{Corolario}
\newtheorem{defi}{Definición}
\newtheorem{lem}{Lema}

\definecolor{td}{RGB}{170,0,0}

\renewcommand\rmdefault{lmr}

%Se definen los encabezados y pies de página
\pagestyle{fancy}
\fancyhf{}
%Encabezados: L: left, R: right, E: even, O: odd
\fancyhead[LE]{Repositorio de Matemáticas}
\fancyhead[RE]{Escuela Colombiana de Ingeniería Julio Garavito}

%~~~~~~~~~~~~MODIFICAR~~~~~~~~~~~~~~~~~~~~~
\fancyhead[RO]{Jornada 2018-I} %~~~~PONER LA CORRESPONDIENTE JORNADA~~~~
\fancyhead[LO]{Ponga el alias de su proyecto aquí} %~~~~PONER EL ALIAS DE SU PROYECTO: <si el título del proyecto es muy largo, debe tener un alias> ~~~~
%~~~~~~~~~~~~~~~~~~~~~~~~~~~~~~~~~~~~~~~~~~~~

\cfoot{\thepage} %pie de página

\begin{document}
\begin{titlepage} %portada
%márgenes para unicamente la portada
\setmargins{2.5cm}{1cm}{16.5cm}{23.42cm}{10pt}{1cm}{0pt}{1cm}

%encabezado portada
\begin{figure}[t]
\begin{minipage}{0.5\textwidth}\large
\begin{flushleft}
\includegraphics[scale=0.5]{ECI.png} %encabezado ECI
\end{flushleft}
\end{minipage}
\begin{minipage}{0.5\textwidth}\large
\begin{flushright}
\includegraphics[scale=0.45]{MATH.png} %encabezado Programa de Matemáticas
\end{flushright}
\end{minipage}
\end{figure}

%~~~~~~~~~~~~MODIFICAR~~~~~~~~~~~~~~~~~~~~~
%En caso de que el título sea muy grande, tratar de hacer más pequeña la letra del mismo con los comandos \small o \tiny \scriptsize \footnotesize 
\title{Ponga aquí el título de su proyecto} %Poner el título de su proyecto
%~~~~~~~~~~~~~~~~~~~~~~~~~~~~~~~~~~~~~~~~~~~~
\date{}
\maketitle
\noindent\begin{tabular}{p{4.5cm} p{11cm}}
\textcolor{td}{\rule{4.5cm}{0.04cm}} 
\par\medskip
\textbf{Proyecto de Semestre}
\par\bigskip
\textbf{Estudiantes:}
\begin{itemize}
%~~~~~~~~~~~~~~~~MODIFICAR~~~~~~~~~~~~~~~~~~~~~~
\item Estudiante 1
\item Estudiante 2
\item Estudiante 3
\item Estudiante 4
%~~~~~~~~~~~~~~~~~~~~~~~~~~~~~~~~~~~~~~~~~~~~~~~
\end{itemize}
\par\medskip
\textbf{Dirigido por:}
\par\medskip \hspace{0.5cm}
%~~~~~~~~~~~~~~~~MODIFICAR~~~~~~~~~~~~~~~~~~~~~~
\textrm{Nombre tutor}
%~~~~~~~~~~~~~~~~~~~~~~~~~~~~~~~~~~~~~~~~~~~~~~~
\par\medskip
\begin{center}
%~~~~~~~~~~~~~~~~MODIFICAR~~~~~~~~~~~~~~~~~~~~~~
\textbf{2018-I}
%~~~~~~~~~~~~~~~~~~~~~~~~~~~~~~~~~~~~~~~~~~~~~~~~
\end{center}
%~~~~~~~~~~~~~~~~~~~~~~~Si es necesario se puede cambiar el espacio que hay entre el logo de la Escuela y el texto de arriba en el vspace <espacio vertical>
\vspace{3cm} 
%~~~~~~~~~~~~~~~~~~~~~~~~~~~~~~~~~~~~~~~~~~~~~~~~~~~~~~~~
\includegraphics[scale=0.5]{logo.png}
&
%~~~~~DE AQUÍ EN ADELANTE CUANDO SE QUIERA INCLUIR TEXTO SE DEBE ELIMINAR UNICAMENTE EL COMANDO "lipsum[#]" y poner su texto correspondiente.
\textbf{Resumen}
\par\medskip
\lipsum[2] %eliminar
\vspace{1cm}
\textbf{Palabras clave}
\par\medskip
Phasellus, donec, eget, turum %eliminar
\par\medskip
\vspace{1cm}
\textbf{Abstract}
\par\medskip
\lipsum[2] %eliminar
\vspace{1cm}
\textbf{Key Words}
\par\medskip
Phasellus, donec, eget, turum %eliminar
\end{tabular}
\end{titlepage}

%Éste es un comando para el índice
\tableofcontents
$\ $

\newpage
\section{Introducción}
\lipsum[1] %eliminar
\lipsum[2] %eliminar

\newpage
\section{Conceptos Preliminares (si los hay)}
\lipsum[1] %eliminar
\subsection{Subsección (si la hay)}
\lipsum %eliminar
\subsection{Subsección (si la hay)}
\lipsum %eliminar

\newpage
\section{Marco Teórico}
\lipsum[1] %eliminar
\subsection{Subsección (si la hay)}
\lipsum %eliminar
\subsection{Subsección (si la hay)}
\lipsum %eliminar

\newpage
\section{Resultados}
\lipsum[1] %eliminar
\subsection{Subsección (si la hay)}
\lipsum %eliminar
\subsection{Subsección (si la hay)}
\lipsum %eliminar

\newpage
\section{Conclusiones}
\lipsum[5] %eliminar
\lipsum[1] %eliminar

\newpage

\begin{thebibliography}{X} %Los siguientes son algunos ejemplos de cita.
\bibitem{LOP} LÓPEZ, F. (SF) \textit{Espacios Topoógicos}, Depto. de Geometría y Topología, Universidad de Granada, E-18071 Granada, España
\bibitem{MACH} MACHO, M. (2002), \textit{¿Qué es la topología?}, Universidad del País Vasco-Euskal Herriko Unibertsitatea.
\bibitem{O'} O’CONNOR, J., ROBERTSON, E. (1996). \textit{A history of Topology.}
\end{thebibliography}
\end{document}

%Esta plantilla fue desarrollada por Tatiana León Zamora, cualquier duda, pueden contactarme: yessica.leon@mail.escuelaing.edu.co

